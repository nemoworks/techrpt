\documentclass[12pt,a4paper]{article}

\usepackage{inputenc}
\usepackage[british,UKenglish]{babel}
\usepackage{amsmath}
%\usepackage{titlesec}
\usepackage{color}
\usepackage{graphicx}
\usepackage{fancyref}
\usepackage{hyperref}
\usepackage{float}
\usepackage{scrextend}
\usepackage{setspace}
\usepackage{xargs}
\usepackage{multicol}
\usepackage{nameref}

\usepackage{sectsty}
\usepackage{multicol}
\usepackage{multirow}
\usepackage[procnames]{listings}
\usepackage{appendix}
\usepackage{geometry}
\usepackage{titlesec}

\newcommand\tab[1][1cm]{\hspace*{#1}}
\hypersetup{colorlinks=true, linkcolor=black}
\interfootnotelinepenalty=10000

\newcommand{\cleancode}[1]{\begin{addmargin}[3em]{3em}\texttt{\textcolor{cleanOrange}{#1}}\end{addmargin}}
\newcommand{\cleanstyle}[1]{\text{\textcolor{cleanOrange}{\texttt{#1}}}}


\usepackage[colorinlistoftodos,prependcaption,textsize=footnotesize]{todonotes}
\newcommandx{\commred}[2][1=]{\textcolor{Red}
{\todo[linecolor=red,backgroundcolor=red!25,bordercolor=red,#1]{#2}}}
\newcommandx{\commblue}[2][1=]{\textcolor{Blue}
{\todo[linecolor=blue,backgroundcolor=blue!25,bordercolor=blue,#1]{#2}}}
\newcommandx{\commgreen}[2][1=]{\textcolor{OliveGreen}{\todo[linecolor=OliveGreen,backgroundcolor=OliveGreen!25,bordercolor=OliveGreen,#1]{#2}}}
\newcommandx{\commpurp}[2][1=]{\textcolor{Plum}{\todo[linecolor=Plum,backgroundcolor=Plum!25,bordercolor=Plum,#1]{#2}}}

\def\code#1{{\tt #1}}

\def\note#1{\noindent{\bf [Note: #1]}}

\makeatletter
%% The "\@seccntformat" command is an auxiliary command
%% (see pp. 26f. of 'The LaTeX Companion,' 2nd. ed.)
\def\@seccntformat#1{\@ifundefined{#1@cntformat}%
   {\csname the#1\endcsname\quad}  % default
   {\csname #1@cntformat\endcsname}% enable individual control
}
\let\oldappendix\appendix %% save current definition of \appendix
\renewcommand\appendix{%
    \oldappendix
    \newcommand{\section@cntformat}{\appendixname~\thesection\quad}
}
\makeatother


% "define" Scala
\usepackage[T1]{fontenc}  
\usepackage[scaled=0.82]{beramono}  
\usepackage{microtype} 

\sbox0{\small\ttfamily A}
\edef\mybasewidth{\the\wd0 }

\lstdefinelanguage{scala}{
  morekeywords={abstract,case,catch,class,def,%
    do,else,extends,false,final,finally,%
    for,if,implicit,import,match,mixin,%
    new,null,object,override,package,%
    private,protected,requires,return,sealed,%
    super,this,throw,trait,true,try,%
    type,val,var,while,with,yield},
  sensitive=true,
  morecomment=[l]{//},
  morecomment=[n]{/*}{*/},
  morestring=[b]",
  morestring=[b]',
  morestring=[b]"""
}

\usepackage{color}
\definecolor{dkgreen}{rgb}{0,0.6,0}
\definecolor{gray}{rgb}{0.5,0.5,0.5}
\definecolor{mauve}{rgb}{0.58,0,0.82}

% Default settings for code listings
\lstset{frame=tb,
  language=scala,
  aboveskip=3mm,
  belowskip=3mm,
  showstringspaces=false,
  columns=fixed, % basewidth=\mybasewidth,
  basicstyle={\small\ttfamily},
  numbers=none,
  numberstyle=\footnotesize\color{gray},
  % identifierstyle=\color{red},
  keywordstyle=\color{blue},
  commentstyle=\color{dkgreen},
  stringstyle=\color{mauve},
  frame=single,
  breaklines=true,
  breakatwhitespace=true,
  procnamekeys={def, val, var, class, trait, object, extends},
  procnamestyle=\ttfamily\color{red},
  tabsize=2
}

\lstnewenvironment{scala}[1][]
{\lstset{language=scala,#1}}
{}
\lstnewenvironment{cpp}[1][]
{\lstset{language=C++,#1}}
{}
\lstnewenvironment{bash}[1][]
{\lstset{language=bash,#1}}
{}
\lstnewenvironment{verilog}[1][]
{\lstset{language=verilog,#1}}
{}



\lstset{frame=, basicstyle={\footnotesize\ttfamily}}
\geometry{left=3.17cm,right=3.17cm,top=2.54cm,bottom=2.54cm}
% 常规页眉页脚
\newpagestyle{main}{            
    \sethead{}{}{}     %设置页眉
    \setfoot{}{\thepage}{}      %设置页脚,可以在页脚添加 \thepage  显示页数
    \headrule                                      % 添加页眉的下划线
    \footrule                                       %添加页脚的下划线
}

% 封面页眉页脚
\newpagestyle{title}{            
    \sethead{报告号: 774323540--10dz1200500/01}{}{}     %设置页眉
%    \setfoot{左页脚}{\thepage}{右页脚}      %设置页脚,可以在页脚添加 \thepage  显示页数
    \headrule                                      % 添加页眉的下划线
    \footrule                                       %添加页脚的下划线
}

\pagestyle{main}

\graphicspath{ {images/} }
\usepackage{ctex}
\usepackage{caption}
\usepackage{subfigure}
%-----------------------------------------BEGIN DOC----------------------------------------

\begin{document}
\renewcommand{\contentsname}{\ \ \ 目录\ \ \ }
\renewcommand{\appendixname}{附录}
\renewcommand{\appendixpagename}{附录}
\renewcommand{\refname}{参考文献} 
\renewcommand{\figurename}{图}
\renewcommand{\tablename}{表}
\renewcommand{\today}{\number\year 年 \number\month 月 \number\day 日}

\renewcommand{\maketitle}{
    \thispagestyle{title}
    \heiti
    \vspace*{94pt}
    \begin{center}
        \fontsize{50pt}{0} 科技报告\\
        \vspace*{300pt}
        \large \textcolor{red}{报告名称: \quad}\ \ \underline{\makebox[300pt]{陈家镇国际生态社区能源管理中心建设关键技术}}\\
        \large 支持渠道: \quad\ \ \underline{\makebox[300pt]{科技部等中央单位与上海市共同推进重大任务科研专项}}\\
        \large 报告类型: \quad\ \ \underline{\makebox[300pt]{最终报告}}\\
        \large 编制单位: \quad\ \ \underline{\makebox[300pt]{上海陈家镇建设发展有限公司}}\\
        \large 编制时间: \quad\ \ \underline{\makebox[300pt]{\today}}

    \end{center}
}

\maketitle

\newpage

%-----------------------------------------ABSTRACT-------------------------------------
\begin{center}
{\Large\bf{摘\ 要\\}}
\end{center}
请在这里输入摘要内容.
\newpage
%-----------------------------------------ABSTRACT-------------------------------------
\begin{center}
{\Large\bf{版\ 权\ 声\ 明\\}}
\end{center}
该文件受《中华人名共和国著作权法》的保护。ERCESI实验室保留拒绝授权违法复制该文件的权利。任何收存和保管本文件各种版本的单位和个人,未经ERCESI实验室(西北工业大学)同意,不得将本文档转借他人,亦不得随意复制、抄录、拍照或以任何方式传播。 否则,引起有碍著作权之问题,将可能承担法律责任。\newpage
%-----------------------------------------CONTENT-------------------------------------
\begin{center}
\tableofcontents\label{c}
\end{center}
\newpage

%------------------------------------------TEXT--------------------------------------------

%----------------------------------------OVERVIEW-----------------------------------------

%----------------------------------SYSTEM DESIGN------------------------------------------

% -----------------------------------BLOCKS DESIGN----------------------------------------

%add more subsections for other block in you CPU design.

\section{基于Istio的协议实现}\label{baseistio}

\newpage
\section{人机物融合平台架构与技术集成} 

\subsection{总体的设计和结构}

\subsection{人机物资源的描述与编排}
\subsubsection{人机物资源描述规范}
\subsubsection{人机物资源的运行时编排}
\subsection{消息驱动的人机物资源协作模型}
\subsubsection{资源状态的同步}
\subsubsection{资源代理服务}
\paragraph{资源代理描述}
\subsubsection{消息队列}

\subsection{面向人机物融合的资源服务治理框架}
\subsubsection{人机物融合场景下的服务路由}
\subsubsection{资源服务的动态更新}
\newpage
\section{动态更新系统实现}\label{}

\subsection{基本设计概念和结构}

\subsection{代码管理器}
\subsubsection{模块描述}
\subsubsection{功能}

\subsection{构建管理器}

\subsection{更新控制器}

% -----------------------------------Appendix----------------------------------------
% -----------------------------------REFERENCE----------------------------------------
\end{document}

