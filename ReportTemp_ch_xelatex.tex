\documentclass[12pt,a4paper]{article}

\input{style/ch_xelatex.tex}
\input{style/scala.tex}

\lstset{frame=, basicstyle={\footnotesize\ttfamily}}
\geometry{left=3.17cm,right=3.17cm,top=2.54cm,bottom=2.54cm}
% 常规页眉页脚
\newpagestyle{main}{            
    \sethead{}{}{}     %设置页眉
    \setfoot{}{\thepage}{}      %设置页脚,可以在页脚添加 \thepage  显示页数
    \headrule                                      % 添加页眉的下划线
    \footrule                                       %添加页脚的下划线
}

% 封面页眉页脚
\newpagestyle{title}{            
    \sethead{报告号: 774323540--10dz1200500/01}{}{}     %设置页眉
%    \setfoot{左页脚}{\thepage}{右页脚}      %设置页脚,可以在页脚添加 \thepage  显示页数
    \headrule                                      % 添加页眉的下划线
    \footrule                                       %添加页脚的下划线
}

\pagestyle{main}

\graphicspath{ {images/} }
\usepackage{ctex}
\usepackage{caption}
\usepackage{subfigure}
%-----------------------------------------BEGIN DOC----------------------------------------

\begin{document}
\renewcommand{\contentsname}{\ \ \ 目录\ \ \ }
\renewcommand{\appendixname}{附录}
\renewcommand{\appendixpagename}{附录}
\renewcommand{\refname}{参考文献} 
\renewcommand{\figurename}{图}
\renewcommand{\tablename}{表}
\renewcommand{\today}{\number\year 年 \number\month 月 \number\day 日}
\songti % set the font for title

\renewcommand{\maketitle}{
    \thispagestyle{title}
    \vspace*{94pt}
    \begin{center}
        \fontsize{50pt}{0} 科技报告\\
        \vspace*{300pt}
        \large \textcolor{red}{项目名称: \quad}\ \ \underline{\makebox[300pt]{待填写}}\\
        \large 课题名称: \quad\ \ \underline{\makebox[300pt]{待填写}}\\
        \large 编制单位: \quad\ \ \underline{\makebox[300pt]{待填写}}\\
        \large 编制时间: \quad\ \ \underline{\makebox[300pt]{\today}}

    \end{center}
}

\maketitle


\songti %set the font for the text
\newpage

%-----------------------------------------ABSTRACT-------------------------------------
\begin{center}
{\Large\bf{摘\ 要\\}}
\end{center}
请在这里输入摘要内容.
\newpage
%-----------------------------------------ABSTRACT-------------------------------------
\begin{center}
{\Large\bf{版\ 权\ 声\ 明\\}}
\end{center}
该文件受《中华人名共和国著作权法》的保护。ERCESI实验室保留拒绝授权违法复制该文件的权利。任何收存和保管本文件各种版本的单位和个人,未经ERCESI实验室(西北工业大学)同意,不得将本文档转借他人,亦不得随意复制、抄录、拍照或以任何方式传播。 否则,引起有碍著作权之问题,将可能承担法律责任。\newpage
%-----------------------------------------CONTENT-------------------------------------
\begin{center}
\tableofcontents\label{c}
\end{center}
\newpage

%------------------------------------------TEXT--------------------------------------------

%----------------------------------------OVERVIEW-----------------------------------------

%----------------------------------SYSTEM DESIGN------------------------------------------

% -----------------------------------BLOCKS DESIGN----------------------------------------

%add more subsections for other block in you CPU design.

\section{基于Istio的协议实现}\label{baseistio}

\newpage
\section{人机物融合平台架构与技术集成} 

\subsection{总体的设计和结构}

\subsection{人机物资源的描述与编排}
\subsubsection{人机物资源描述规范}
\subsubsection{人机物资源的运行时编排}
\subsection{消息驱动的人机物资源协作模型}
\subsubsection{资源状态的同步}
\subsubsection{资源代理服务}
\paragraph{资源代理描述}
\subsubsection{消息队列}

\subsection{面向人机物融合的资源服务治理框架}
\subsubsection{人机物融合场景下的服务路由}
\subsubsection{资源服务的动态更新}
\newpage
\section{动态更新系统实现}\label{}

\subsection{基本设计概念和结构}

\subsection{代码管理器}
\subsubsection{模块描述}
\subsubsection{功能}

\subsection{构建管理器}

\subsection{更新控制器}

% -----------------------------------Appendix----------------------------------------
% -----------------------------------REFERENCE----------------------------------------
\end{document}

